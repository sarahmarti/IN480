\singlespacing

\chapter{Implementazione}

\section{csrCreate, brc2Coo, coo2Csr, triples2mat}

\texttt{csrCreate}: effettua la conversione dal formato BRC (Binary Row Compressed) al formato CSR (Compressed Sparse Row), utilizzando due modi 
differenti a seconda se la dimensione della matrice � sconosciuta (\texttt{shape=(0,0)}) o no. \\
% Conversion from BRC to CSR format. The transformation from BRC to CSR format is implemented slightly differently, according to the fact that the 
% matrix dimension is either unknown (shape=(0,0)) or known

\noindent \texttt{brc2Coo}: trasforma una matrice in rappresentazione BRC in una lista di triple (\emph{riga, colonna, 1}), ordinate per riga. \\
% From triples to scipy.sparse The function brc2Coo transforms a BRC representation in a list of triples (row, column, 1) ordered by row.

\noindent \texttt{coo2Csr}: effettua la conversione dalle triple al formato CSR. \\
%The conversion from triples to csr format is provided below.

\noindent \texttt{triples2mat}: realizza la trasformazione della matrice in formato \emph{sparso}, data come lista di triple (\emph{riga, colonna, 
valore}) di elementi non nulli, nel formato \texttt{scipy.sparse} corrispondente al parametro \texttt{shape} in ingresso, settato di default a 
\texttt{csr}, il formato standard dell'infrastruttura \texttt{larcc}.
% Conversion to csr format Then we give the function triples2mat to make the transformation from the sparse matrix, given as a list of triples 
% row,column,value (non-zero elements), to the scipy.sparse format corresponding to the shape parameter, set by default to "csr", that stands for 
% Compressed Sparse Row, the normal matrix format of the LARCC framework.

\subsection{Traduzione}

\textbf{Python}

\begin{codice}
def triples2mat(triples, shape = "csr"):
    n = len(triples)
    data = arange(n)
    ij = arange(2*n).reshape(2,n)
    for k, item in enumerate(triples):
	ij[0][k], ij[1][k], data[k] = item
    return scipy.sparse.coo_matrix((data,ij)).asformat(shape)
\end{codice}

\vspace{0.5 cm}

\begin{codice}
def brc2Coo(ListOfListOfInt):
    COOm = [[k,col,1] for k,row in enumerate(ListOfListOfInt) for col in row]
    return COOm
\end{codice}

\vspace{0.5 cm}

\begin{codice}
def coo2Csr(COOm):
    CSRm = triples2mat(COOm, "csr")
    return CSRm
\end{codice}
  
\vspace{0.5 cm}

\begin{codice}
def csrCreate(BRCmatrix, lenV = 0, shape = (0,0)):
    triples = brc2Coo(BRCmatrix)
    if shape == (0,0):
	CSRmatrix = coo2Csr(triples)
    else:
	CSRmatrix = scipy.sparse.csr_matrix(shape)
	for i,j,v in triples: 
	    CSRmatrix[i,j] = v
    return CSRmatrix
\end{codice}

\vspace{0.5 cm}

\noindent \textbf{Julia}

\begin{codice}
function triples2mat(triples, shape = "csr")
    n = length(triples)
    data = collect(1:n)
    i = collect(0:n-1)
    j = collect(n:2n-1)
    ij = [i,j]
    for e in enumerate(triples)
	k = e[1]
	ij[1][k] = triples[k][1]
	ij[2][k] = triples[k][2]
	data[k] = triples[k][3]
    end
    return ss.coo_matrix((data,ij))[:asformat](shape)
end
\end{codice}

\vspace{0.5 cm}

\begin{codice}
function brc2Coo(BRCmatrix)
    COOmatrix = vcat([[k-1,col,1] for (k,row) in enumerate(BRCmatrix) 
		for col in row])
    return COOmatrix
end
\end{codice} 

\vspace{0.5 cm}

\begin{codice}
function coo2Csr(COOmatrix)
    CSRmatrix = triples2mat(COOmatrix, "csr")
    return CSRmatrix
end
\end{codice}

\vspace{0.5 cm}

\begin{codice}
function csrCreate(BRCmatrix, lengthV = 0, shape = (0,0))
    triples = brc2Coo(BRCmatrix)
    if shape == (0,0)
	CSRmatrix = coo2Csr(triples)
    else
	CSRmatrix = ss.csr_matrix(shape)
	for (i,j,v) in triples
	    CSRmatrix[i,j] = v
	end
    end
    return CSRmatrix 
end 
\end{codice}

\vspace{0.5 cm}

\subsection{Versione parallela}

\begin{codice}
@everywhere function ptriples2mat(triples, shape = "csr")
    n = length(triples)
    data = collect(1:n)
    i = collect(0:n-1)
    j = collect(n:2n-1)
    ij = [i,j]
    @sync for e in enumerate(triples)
	k = e[1]
	ij[1][k] = triples[k][1]
	ij[2][k] = triples[k][2]
	data[k] = triples[k][3]
    end
    return ss.coo_matrix((data,ij))[:asformat](shape)
end
\end{codice}

\vspace{0.5 cm}

\begin{codice}
@everywhere function pbrc2Coo(BRCmatrix)
    COOmatrix = vcat([[k-1,col,1] for (k,row) in enumerate(BRCmatrix) 
		for col in row])
    return COOmatrix
end
\end{codice}

\vspace{0.5 cm}

\begin{codice}
@everywhere function pcoo2Csr(COOmatrix)
    CSRmatrix = ptriples2mat(COOmatrix, "csr")
    return CSRmatrix
end
\end{codice}

\vspace{0.5 cm}

\begin{codice}
@everywhere function pcsrCreate(BRCmatrix, lengthV = 0, shape = (0,0))
    triples = pbrc2Coo(BRCmatrix)
    if shape == (0,0)
	CSRmatrix = pcoo2Csr(triples)
    else
	CSRmatrix = ss.csr_matrix(shape)
	@sync for (i,j,v) in triples
	    CSRmatrix[i,j] = v
	end
    end
    return CSRmatrix 
end 
\end{codice}

\vspace{0.5 cm}

\subsection{Tempi di esecuzione}

\noindent \textbf{Seriale}

\begin{codice}
julia> @timev csrCreate(EV)
  0.001009 seconds (156 allocations: 8.359 KiB)
elapsed time (ns): 1009260
bytes allocated:   8560
pool allocs:       156
PyObject <9x6 sparse matrix of type '<type 'numpy.int64'>'
	with 18 stored elements in Compressed Sparse Row format>
\end{codice}

\vspace{0.5 cm}

\noindent \textbf{Parallela con lo stesso numero di processori}

\begin{codice}
julia> @timev pcsrCreate(EV)
  0.000569 seconds (156 allocations: 8.359 KiB)
elapsed time (ns): 569140
bytes allocated:   8560
pool allocs:       156
PyObject <9x6 sparse matrix of type '<type 'numpy.int64'>'
	with 18 stored elements in Compressed Sparse Row format
\end{codice}

\vspace{0.5 cm}

\pagebreak

\noindent \textbf{Parallela con l'aggiunta di 15 processori}

\begin{codice}
julia> @timev pcsrCreate(EV)
  0.002283 seconds (156 allocations: 8.359 KiB)
elapsed time (ns): 2282594
bytes allocated:   8560
pool allocs:       156
PyObject <9x6 sparse matrix of type '<type 'numpy.int64'>'
	with 18 stored elements in Compressed Sparse Row format>
\end{codice}


\section{csr2DenseMatrix, csrGetNumberOfRows, csrGetNumberOfColumns}

\texttt{csr2DenseMatrix}: il pacchetto \texttt{scipy} fornisce l'utile metodo \texttt{.todense()} allo scopo di trasformare ogni matrice nel formato 
\emph{sparso} nel corrispondente formato \emph{denso}. Per ragioni di generalit� e portabilit� � riportata la funzione \emph{csr2DenseMatrix}. \\
% Sparse to dense matrix transformation The Scipy package provides the useful method .todense() in order to transform any sparse matrix format in the 
% corresponding dense format. The function csr2DenseMatrix is given here for the sake of generality and portability.

\noindent \texttt{csrGetNumberOfRows}, \texttt{csrGetNumberOfColumns}: le due funzioni ausiliarie consentono di chiedere il numero di righe e colonne 
di una matrice CSR, indipendentemente dall'implementazione a basso livello (che in questo caso � fornita da \texttt{scipy.sparse}).
% Two utility functions allow to query the number of rows and columns of a CSR matrix, independently from the low-level implementation (that in the 
% following is provided by scipy.sparse).

\subsection{Traduzione}

\textbf{Python}

\begin{codice}
def csrGetNumberOfRows(CSRmatrix):
    Int = CSRmatrix.shape[0]    
    return Int
\end{codice}

\vspace{0.5 cm}

\begin{codice}
def csrGetNumberOfColumns(CSRmatrix):
    Int = CSRmatrix.shape[1]
    return Int
\end{codice}

\vspace{0.5 cm}

\begin{codice}
def csr2DenseMatrix(CSRm):
    nrows = csrGetNumberOfRows(CSRm)
    ncolumns = csrGetNumberOfColumns(CSRm)
    ScipyMat = zeros((nrows, ncolumns), int)
    C = CSRm.tocoo()
    for triple in zip(C.row, C.col, C.data):
	ScipyMat[triple[0], triple[1]] = triple[2]
    return ScipyMat
\end{codice}

\noindent \textbf{Julia}

\begin{codice}
function csrGetNumberOfRows(CSRmatrix)
    shape = CSRmatrix[:get_shape]()
    Int = shape[1]
    return Int
end
\end{codice}

\vspace{0.5 cm}

\begin{codice}
function csrGetNumberOfColumns(CSRmatrix)
    shape = CSRmatrix[:get_shape]()
    Int = shape[2]
    return Int
end
\end{codice}

\vspace{0.5 cm}

\begin{codice}
function csr2DenseMatrix(CSRmatrix)
    nrows = csrGetNumberOfRows(CSRmatrix)
    ncolumns = csrGetNumberOfColumns(CSRmatrix)
    ScipyMat = zeros(Int64, (nrows, ncolumns))
    C = CSRmatrix[:tocoo]()
    for triple in zip(C[:row], C[:col], C[:data])
	ScipyMat[triple[1]+1, triple[2]+1] = triple[3]
    end
    return ScipyMat
end
\end{codice}

\vspace{0.5 cm}

\subsection{Versione parallela}

\begin{codice}
@everywhere function pcsrGetNumberOfRows(CSRmatrix)
    shape = CSRmatrix[:get_shape]()
    Int = shape[1]
    return Int
end 
\end{codice}

\vspace{0.5 cm}

\begin{codice}
@everywhere function pcsrGetNumberOfColumns(CSRmatrix)
    shape = CSRmatrix[:get_shape]()
    Int = shape[2]
    return Int
end 
\end{codice}

\vspace{0.5 cm}

\begin{codice}
@everywhere function pcsr2DenseMatrix(CSRmatrix)
    nrows = pcsrGetNumberOfRows(CSRmatrix)
    ncolumns = pcsrGetNumberOfColumns(CSRmatrix)
    ScipyMat = zeros(Int64, (nrows, ncolumns))
    C = CSRmatrix[:tocoo]()
    @sync for triple in zip(C[:row], C[:col], C[:data])
	ScipyMat[triple[1]+1, triple[2]+1] = triple[3]
    end
    return ScipyMat
end
\end{codice}

\vspace{0.5 cm}

\subsection{Tempi di esecuzione}

\noindent \textbf{Seriale}

\begin{codice}
julia> @timev csr2DenseMatrix(csrEV)
  0.001151 seconds (326 allocations: 13.125 KiB)
elapsed time (ns): 1150963
bytes allocated:   13440
pool allocs:       326
9x6 Array{Int64,2}:
 1  1  0  0  0  0
 1  0  0  1  0  0
 0  1  1  0  0  0
 0  1  0  1  0  0
 0  1  0  0  1  0
 0  0  1  0  1  0
 0  0  1  0  0  1
 0  0  0  1  1  0
 0  0  0  0  1  1 
\end{codice}

\vspace{0.5 cm}

\noindent \textbf{Parallela con lo stesso numero di processori}

\begin{codice}
julia> @timev pcsr2DenseMatrix(pcsrEV)
  0.001471 seconds (330 allocations: 13.313 KiB)
9x6 Array{Int64,2}:
 1  1  0  0  0  0
 1  0  0  1  0  0
 0  1  1  0  0  0
 0  1  0  1  0  0
 0  1  0  0  1  0
 0  0  1  0  1  0
 0  0  1  0  0  1
 0  0  0  1  1  0
 0  0  0  0  1  1  
\end{codice}

\vspace{0.5 cm}

\noindent \textbf{Parallela con l'aggiunta di 15 processori}

\begin{codice}
julia> @timev pcsr2DenseMatrix(pcsrEV)
  0.046019 seconds (330 allocations: 13.313 KiB)
9x6 Array{Int64,2}:
 1  1  0  0  0  0
 1  0  0  1  0  0
 0  1  1  0  0  0
 0  1  0  1  0  0
 0  1  0  0  1  0
 0  0  1  0  1  0
 0  0  1  0  0  1
 0  0  0  1  1  0
 0  0  0  0  1  1 
\end{codice}


\section{larModelNumbering}

\texttt{larModelNumbering}: realizza la visualizzazione \emph{numerata} di un modello LAR, utilizzando gli indici delle celle che vengono colorati 
in 4 modi differenti.
% Visualization of cell indices 20b i ?
% """ Numbered visualization of a LAR model """

\subsection{Traduzione}

\textbf{Python}

\begin{codice}
def larModelNumbering(scalx = 1, scaly = 1, scalz = 1):
    def larModelNumbering0(V, bases, submodel, numberScaling = 1):
	color = [ORANGE, CYAN, GREEN, WHITE]
	nums = AA(range)(AA(len)(bases))
	hpcs = [submodel]
	for k in range(len(bases)):
	    hpcs += [cellNumbering((V, bases[k]), submodel)
		    (nums[k], color[k], (0.5+0.1*k)*numberScaling)]
	return STRUCT(hpcs)
    return larModelNumbering0
\end{codice}

\vspace{0.5 cm}

\noindent \textbf{Julia}

\begin{codice}
function larModelNumbering(scalx = 1, scaly = 1, scalz = 1)
    function larModelNumbering0(V, bases, submodel, numberScaling = 1)
	color = [p.ORANGE, p.CYAN, p.GREEN, p.WHITE]
	nums = [collect(0:length(bases[1])), 
		collect(0:length(bases[2])), 
		collect(0:length(bases[3]))]
	hpcs = [submodel]
	for k in collect(1:length(bases))
	    hpcs = vcat(hpcs, [l.cellNumbering((V,bases[k]), submodel)
			      (nums[k], color[k], (0.5+0.1*(k-1))*numberScaling)])
	end
	return p.STRUCT(hpcs)
    end
    return larModelNumbering0
end
\end{codice}

\vspace{0.5 cm}

\subsection{Versione parallela}

\begin{codice}
@everywhere function plarModelNumbering(scalx=1,scaly=1,scalz=1)
    function plarModelNumbering0(V,bases,submodel,numberScaling=1)
	color = [p.ORANGE,p.CYAN,p.GREEN,p.WHITE]
	nums = [collect(0:length(bases[1])),
		collect(0:length(bases[2])),
		collect(0:length(bases[3]))]
	hpcs = [submodel]
	@sync for k in collect(1:length(bases))
	    hpcs = vcat(hpcs, [l.cellNumbering((V,bases[k]),submodel)
			      (nums[k],color[k],(0.5+0.1*(k-1))*numberScaling)])
	end
	return p.STRUCT(hpcs)
    end
    return plarModelNumbering0
end 
\end{codice}

\vspace{0.5 cm}

\subsection{Tempi di esecuzione}

\noindent \textbf{Seriale}

\begin{codice}
julia> @timev larModelNumbering(1,1,1)(V,[VV,EV,FV],submodel,2)
centre
centre
centre
  0.068753 seconds (1.75 k allocations: 95.000 KiB)
elapsed time (ns): 68752973
bytes allocated:   97280
pool allocs:       1752
PyObject <pyplasm.xgepy.Hpc; proxy of <Swig Object of type 'std::shared_ptr< Hpc > *' 
at 0x7f7f2f52c390> >
\end{codice}

\vspace{0.5 cm}

\noindent \textbf{Parallela con lo stesso numero di processori}

\begin{codice}
julia> @timev plarModelNumbering(1,1,1)(V,[VV,EV,FV],submodel,2)
centre
centre
centre
  0.070618 seconds (2.36 k allocations: 172.938 KiB)
elapsed time (ns): 70617629
bytes allocated:   177088
pool allocs:       2360
non-pool GC allocs:1
malloc() calls:    2
PyObject <pyplasm.xgepy.Hpc; proxy of <Swig Object of type 'std::shared_ptr< Hpc > *' 
at 0x7f7f2f40bc30> >
\end{codice}

\vspace{0.5 cm}

\pagebreak 

\noindent \textbf{Parallela con l'aggiunta di 15 processori}

\begin{codice}
julia> @timev plarModelNumbering(1,1,1)(V,[VV,EV,FV],submodel,2)
centre
centre
centre
  0.070363 seconds (1.76 k allocations: 95.188 KiB)
elapsed time (ns): 70363111
bytes allocated:   97472
pool allocs:       1756
PyObject <pyplasm.xgepy.Hpc; proxy of <Swig Object of type 'std::shared_ptr< Hpc > *' 
at 0x7f7f30fad5a0> >
\end{codice}

\section{larFacets, setup}

\texttt{larFacets}: produce l'estrazione delle facce di un complesso di celle. Restituisce il modello LAR \texttt{V,cellFacets} partendo dal parametro 
\texttt{model} in input. Due parametri opzionali definiscono la dimensione (intrinseca) delle celle in input, con valore di default uguale a 3, e 
l'eventuale presenza di un numero di celle vuote (\texttt{emptyCellNumber}). Il numero � zero di default quando il complesso � chiuso, ad esempio nel 
caso del d-bordo di un (d + 1)-complesso. Se sono presenti celle vuote, il loro sottoinsieme deve essere collocato alla fine della lista di celle.
% Extraction of facets of a cell complex The following larFacets function returns the LAR model V,cellFacets starting from the input model parameter. 
% Two optional parameters define the (intrinsic) dimension of the input cells, with default value equal to three, and the eventual presence of a 
% emptyCellNumber of empty cells. Their number default to zero when the complex is closed, for example in the case it provides the d-boundary of a 
% (d + 1)-complex. If empty cells are present, their subset must be located at the end of the cell list.

%  """ Estraction of (d-1)-cellFacets 
%      from "model" := (V,d-cells)
%      Return (V, (d-1)-cellFacets)
%      """

\subsection{Traduzione}

\textbf{Python}

\begin{codice}
def setup(model, dim):
    V, cells = model
    csr = csrCreate(cells)
    csrAdjSquareMat = larCellAdjacencies(csr)
    csrAdjSquareMat = csrPredFilter(csrAdjSquareMat, GE(dim))
    return V, cells, csr, csrAdjSquareMat
\end{codice}

\vspace{0.5 cm}

\begin{codice}
def larFacets(model, dim = 3, emptyCellNumber = 0):
    V, cells, csr, csrAdjSquareMat = setup(model, dim)
    solidCellNumber = len(cells) - emptyCellNumber
    cellFacets = []
    # for each input cell i
    for i in range(len(cells)):
	adjCells = csrAdjSquareMat[i].tocoo()
	cell1 = csr[i].tocoo().col
	pairs = zip(adjCells.col, adjCells.data)
	for j,v in pairs:
	    if (i < j) and (i < solidCellNumber):
		cell2 = csr[j].tocoo().col
		cell = list(set(cell1).intersection(cell2))
		cellFacets.append(sorted(cell))
    # sort and remove duplicates
    cellFacets = sorted(AA(list)(set(AA(tuple)(cellFacets))))
    return V, cellFacets
\end{codice}

\vspace{0.5 cm}

\noindent \textbf{Julia}

\begin{codice}
function setup(model, dim)
    V, cells = model
    csr = csrCreate(cells)
    csrAdjSquareMat = larCellAdjacencies(csr)
    csrAdjSquareMat = csrPredFilter(csrAdjSquareMat, dim)
    return V, cells, csr, csrAdjSquareMat
end
\end{codice} 

\vspace{0.5 cm}

\begin{codice}
function larFacets(model, dim=3, emptyCellNumber=0)
    V,cells,csr,csrAdjSquareMat = setup(model,dim)
    solidCellNumber = length(cells) - emptyCellNumber
    cellFacets = []
    for i in collect(1:length(cells))
	adjCells = csrAdjSquareMat[i][:tocoo]()
	cell1 = csr[i][:tocoo]()[:col]
	pairs = zip(adjCells[:col],adjCells[:data])
        for pair in pairs
	    if (i<pair[1]+1) && (i<solidCellNumber+1)
		cell2 = csr[pair[1]+1][:tocoo]()[:col]
		cell = intersect(cell1,cell2)				
		cellFacets = append!(cellFacets,[sort(cell)])
	    end    
	end
    end
    # sort and remove duplicates
    t = []
    # crea una lista t di tuple, dopo aver creato l'INSIEME a partire da cellFacets
    for e in Set(cellFacets)
	t = vcat(t, ntuple(i -> e[i],length(e)))
    end
    # crea un array associando ad ogni elemento di t la sua posizione in ordine crescente 
    sp = sortperm(t)
    # crea una lista ordinata t seguendo le posizioni degli elementi
    t = t[sp]
    list = []
    # trasforma la lista di tuple in una lista di liste
    for elem in t
	list = vcat(list, [[elem[1],elem[2]]])
    end
    return V,list
end
\end{codice}

\vspace{0.5 cm}

\subsection{Versione parallela}

\begin{codice}
@everywhere function psetup(model,dim)
    V, cells = model
    csr = csrCreate(cells)
    csrAdjSquareMat = plarCellAdjacencies(csr)
    csrAdjSquareMat = pcsrPredFilter(csrAdjSquareMat,dim)
    return V,cells,csr,csrAdjSquareMat
end 
\end{codice}

\vspace{0.5 cm}

\begin{codice}
@everywhere function plarFacets(model, dim=3, emptyCellNumber=0)
    V,cells,csr,csrAdjSquareMat = psetup(model,dim)
    solidCellNumber = length(cells) - emptyCellNumber
    cellFacets = []
    @sync for i in collect(1:length(cells))
	adjCells = csrAdjSquareMat[i][:tocoo]()
	cell1 = csr[i][:tocoo]()[:col]
	pairs = zip(adjCells[:col],adjCells[:data])
	@sync for pair in pairs
	    if (i<pair[1]+1) && (i<solidCellNumber+1)
		cell2 = csr[pair[1]+1][:tocoo]()[:col]
		cell = intersect(cell1,cell2)				
		cellFacets = append!(cellFacets,[sort(cell)])
	    end    
	end
    end
    # sort and remove duplicates
    t = []
    # crea una lista t di tuple, dopo aver creato l'INSIEME a partire da cellFacets
    @sync for e in Set(cellFacets)
	t = vcat(t, ntuple(i -> e[i],length(e)))
    end
    # crea un array associando ad ogni elemento di t la sua posizione in ordine crescente 
    sp = sortperm(t)
    # crea una lista ordinata t seguendo le posizioni degli elementi
    t = t[sp]
    list = []
    # trasforma la lista di tuple in una lista di liste
    @sync for elem in t
	list = vcat(list, [[elem[1],elem[2]]])
    end
    return V,list
end
\end{codice}

\vspace{0.5 cm}

\subsection{Tempi di esecuzione}

\noindent \textbf{Seriale}

\begin{codice}
julia> @timev larFacets(model,2,1)
  0.210991 seconds (44.40 k allocations: 2.133 MiB)
elapsed time (ns): 210990828
bytes allocated:   2236952
pool allocs:       44395
non-pool GC allocs:2
(Array{Int64,1}[[9, 0], [13, 2], [15, 4], [17, 8], ...)
\end{codice}

\vspace{0.5 cm}

\noindent \textbf{Parallela con lo stesso numero di processori}

\begin{codice}
julia> @timev plarFacets(model,2,1)
  0.212279 seconds (44.56 k allocations: 2.142 MiB)
elapsed time (ns): 212278970
bytes allocated:   2245624
pool allocs:       44563
non-pool GC allocs:2
(Array{Int64,1}[[9, 0], [13, 2], [15, 4], [17, 8], ...)
\end{codice}

\vspace{0.5 cm}

\noindent \textbf{Parallela con l'aggiunta di 15 processori}

\begin{codice}
julia> @timev plarFacets(model,2,1)
  0.234690 seconds (45.36 k allocations: 2.154 MiB, 3.99% gc time)
elapsed time (ns): 234690376
gc time (ns):      9372822
bytes allocated:   2258840
pool allocs:       45362
non-pool GC allocs:2
GC pauses:         1
(Array{Int64,1}[[9, 0], [13, 2], [15, 4], [17, 8], ...)
\end{codice}


\section{larCellAdjacencies, matrixProduct, csrTranspose}

\texttt{larCellAdjacencies}: effettua il calcolo delle celle adiacenti. \\

\noindent Le funzioni seguenti costituiscono le interfacce per due importanti operazioni sulle matrici richieste dal modulo \texttt{larcc}.\\

\noindent \texttt{matrixProduct}: calcola il prodotto binario di matrici compatibili. \\

\noindent \texttt{csrTranspose}: calcola la trasposta di una matrice.

% Matrix product and transposition The following macro provides the IDE interface for the two main matrix operations required by LARCC, the binary 
% product of compatible matrices and the unary transposition of matrices.

\subsection{Traduzione}

\textbf{Python}

\begin{codice}
def csrTranspose(CSRm):
    CSRm = CSRm.T
    return CSRm
\end{codice}

\vspace{0.5 cm}

\begin{codice}
def matrixProduct(CSRm1, CSRm2):
    CSRm = CSRm1 * CSRm2
    return CSRm
\end{codice}

\vspace{0.5 cm}

\begin{codice}
def larCellAdjacencies(CSRm):
    CSRm = matrixProduct(CSRm, csrTranspose(CSRm))
    return CSRm
\end{codice}

\vspace{0.5 cm}

\noindent \textbf{Julia}

\begin{codice}
function csrTranspose(CSRmatrix)
    return CSRmatrix[:transpose]()
end
\end{codice}

\vspace{0.5 cm}

\begin{codice}
function matrixProduct(CSRm1, CSRm2)
    CSRm = CSRm1 * CSRm2
    return CSRm
end
\end{codice} 

\vspace{0.5 cm}

\begin{codice}
function larCellAdjacencies(CSRm)
    CSRm = matrixProduct(CSRm, csrTranspose(CSRm))
    return CSRm
end
\end{codice}

\vspace{0.5 cm}

\subsection{Versione parallela}

\begin{codice}
@everywhere function pcsrTranspose(CSRmatrix)
    return CSRmatrix[:transpose]()
end
\end{codice}

\vspace{0.5 cm}

\begin{codice}
@everywhere function pmatrixProduct(CSRm1,CSRm2)
    CSRm = CSRm1 * CSRm2
    return CSRm
end 
\end{codice}

\vspace{0.5 cm}

\begin{codice}
@everywhere function plarCellAdjacencies(CSRm)
    CSRm = pmatrixProduct(CSRm,csrTranspose(CSRm))
    return CSRm
end
\end{codice}


\section{csrPredFilter, check}

\texttt{csrPredFilter}: nell'implementazione di vari operatori topologici, alcune operazioni sugli elementi delle matrici sono necessarie. Questa 
funzione implementa un'operazione di filtro su una matrice, tramite un \emph{predicato} generico. \\

\noindent \texttt{check}: funzione ausiliaria utilizzata da \texttt{csrPredFilter}, implementata solo in Julia.

% Matrix elements filtering Some filtering operations on matrix elements are needed in the implementation of various topological operators. Some of 
% such filtering operations are given below.
% Matrix filtering via a generic predicate 

\subsection{Traduzione}

\textbf{Python}

\begin{codice}
def csrPredFilter(CSRm, pred):
    coo = CSRm.tocoo()
    triples = [[row,col,val] for row,col,val in zip(coo.row, coo.col, coo.data) 
		if pred(val)]
    i, j, data = TRANS(triples)
    CSRm = scipy.sparse.coo_matrix((data,(i,j)),CSRm.shape).tocsr()
    return CSRm
\end{codice}

\vspace{0.5 cm}

\noindent \textbf{Julia}

\begin{codice}
function check(x, dim)
    return x >= dim
end
\end{codice}

\vspace{0.5 cm}

\begin{codice}
function csrPredFilter(CSRm, dim)
    triples = []
    i = []
    j = []
    data = []
    coo = CSRm[:tocoo]()
    for z in zip(coo[:row], coo[:col], coo[:data])
	if check(z[3], dim)
	    triples = vcat(triples, [[z[1],z[2],z[3]]])
	end
    end
    for t in triples
	i = vcat(i,t[1])
	j = vcat(j,t[2])
	data = vcat(data,t[3])
    end
    CSRm = ss.coo_matrix((data,(i,j)),CSRm[:shape])[:tocsr]()
    return CSRm
end
\end{codice} 

\vspace{0.5 cm}

\subsection{Versione parallela}

\begin{codice}
@everywhere function pcheck(x,dim)
    return x >= dim
end 
\end{codice}

\vspace{0.5 cm}

\begin{codice}
@everywhere function pcsrPredFilter(CSRm, dim)
    triples = []
    i = []
    j = []
    data = []
    coo = CSRm[:tocoo]()
    @sync for z in zip(coo[:row],coo[:col],coo[:data])
	if pcheck(z[3],dim)
	    triples = vcat(triples, [[z[1],z[2],z[3]]])
	end
    end
    for t in triples
	i = vcat(i,t[1])
        j = vcat(j,t[2])
        data = vcat(data,t[3])
    end
    CSRm = ss.coo_matrix((data,(i,j)),CSRm[:shape])[:tocsr]()
    return CSRm
end
\end{codice}


\section{mkSignedEdges}

\texttt{mkSignedEdges}: realizza il disegno di archi orientati, restituendo l'hpc del disegno con le frecce delle 1-celle orientate di un complesso 
cellulare 2D. L'orientamento di ogni arco va dal secondo al primo vertice, indipendentemente dagli indici dei vertici stessi. Pertanto, l'orientamento 
degli archi pu� essere invertito scambiando gli indici dei vertici nella definizione della 1-cella.
% Drawing of oriented edges The following function return the hpc of the drawing with arrows of the oriented 1-cells of a 2D cellular complex. Of 
% course, each edge orientation is from second to first vertex, independently from the vertex indices. Therefore, the edge orientation can be reversed 
% by swapping the vertex indices in the 1-cell definition.

\subsection{Traduzione}

\textbf{Python}

\begin{codice}
def mkSignedEdges(model, scalingFactor = 1):
    V,EV = model
    assert len(V[0]) == 2
    hpcs = []
    times = C(SCALARVECTPROD)
    frac = 0.06*scalingFactor
    for e0,e1 in EV:
        v0,v1 = V[e0],V[e1]
        vx,vy = DIFF([v1,v0])
        nx,ny = [-vy,vx]
        v2 = SUM([v0, times(0.66)([vx,vy])])
        v3 = SUM([v0, times(0.66-frac)([vx,vy]), times(frac)([nx,ny])])
        v4 = SUM([v0, times(0.66-frac)([vx,vy]), times(-frac)([nx,ny])])
        verts,cells = [v0,v1,v2,v3,v4],[[1,2],[3,4],[3,5]]
        hpcs += [MKPOL([verts, cells, None])]
	print(verts)
	print(cells)
    hpc = STRUCT(hpcs)
    return hpc
\end{codice}

\vspace{0.5 cm}

\noindent \textbf{Julia}

\begin{codice}
function mkSignedEdges(model,scalingFactor=1)
    V,EV = model
    assert(length(V[1])==2)
    hpcs = []
    frac = 0.06*scalingFactor
    for e in EV
	# cambio gli indici per farli partire da 1 (siamo in Julia)
	e1 = e[1]+1
	e2 = e[2]+1
	v1 = V[e1]
	v2 = V[e2]
	vx = v2[1]-v1[1]
	vy = v2[2]-v1[2]
	nx = -vy
	ny = vx
	v3 = v1 + (0.66*[vx,vy])
	v4 = v1 + ((0.66-frac)*[vx,vy]) + (frac*[nx,ny])
	v5 = v1 + ((0.66-frac)*[vx,vy]) + (-frac*[nx,ny])
	verts = [v1,v2,v3,v4,v5]
	cells = [[1,2],[3,4],[3,5]]
	W = [Any[vert[h] for h=1:length(vert)] for vert in verts]
	CW = [Any[cell[h] for h=1:length(cell)] for cell in cells]
	hpcs = vcat(hpcs, p.MKPOL(PyObject([W,CW,[]])))
    end
    hpc = p.STRUCT(hpcs)
    return hpc
end
\end{codice}

\vspace{0.5 cm}

\subsection{Versione parallela}

\begin{codice}
@everywhere function pmkSignedEdges(model,scalingFactor=1)
    V,EV = model
    assert(length(V[1])==2)
    hpcs = []
    frac = 0.06*scalingFactor
    @sync for e in EV
	# cambio gli indici per farli partire da 1 (siamo in Julia)
	e1 = e[1]+1
	e2 = e[2]+1
	v1 = V[e1]
	v2 = V[e2]
	vx = v2[1]-v1[1]
	vy = v2[2]-v1[2]
	nx = -vy
	ny = vx
	v3 = v1 + (0.66*[vx,vy])
	v4 = v1 + ((0.66-frac)*[vx,vy]) + (frac*[nx,ny])
	v5 = v1 + ((0.66-frac)*[vx,vy]) + (-frac*[nx,ny])
	verts = [v1,v2,v3,v4,v5]
	cells = [[1,2],[3,4],[3,5]]
	W = [Any[cell[h] for h=1:length(cell)] for cell in verts]
	CW = [Any[cell[h] for h=1:length(cell)] for cell in cells]
	hpcs = vcat(hpcs, p.MKPOL(PyObject([W,CW,[]])))
    end
    hpc = p.STRUCT(hpcs)
    return hpc
end 
\end{codice}

\vspace{0.5 cm}

\subsection{Tempi di esecuzione}

\noindent \textbf{Seriale}

\begin{codice}
julia> @timev mkSignedEdges((V,EV))
  0.033148 seconds (11.85 k allocations: 484.578 KiB)
elapsed time (ns): 33148194
bytes allocated:   496208
pool allocs:       11850
PyObject <pyplasm.xgepy.Hpc; proxy of <Swig Object of type 'std::shared_ptr< Hpc > *' 
at 0x7f7f30ff9510> >
\end{codice}

\vspace{0.5 cm}

\noindent \textbf{Parallela con lo stesso numero di processori}

\begin{codice}
julia> @timev pmkSignedEdges((V,EV))
  0.032413 seconds (11.85 k allocations: 484.766 KiB)
elapsed time (ns): 32412705
bytes allocated:   496400
pool allocs:       11854
PyObject <pyplasm.xgepy.Hpc; proxy of <Swig Object of type 'std::shared_ptr< Hpc > *' 
at 0x7f7f2f3efdb0> >
\end{codice}

\vspace{0.5 cm}

\noindent \textbf{Parallela con l'aggiunta di 15 processori}

\begin{codice}
julia> @timev pmkSignedEdges((V,EV))
  0.025127 seconds (11.85 k allocations: 484.766 KiB)
elapsed time (ns): 25126705
bytes allocated:   496400
pool allocs:       11854
PyObject <pyplasm.xgepy.Hpc; proxy of <Swig Object of type 'std::shared_ptr< Hpc > *' 
at 0x7f7f31019720> >
\end{codice}

\section{Test}

In questa sezione si riporta la traduzione da Python a Julia dei file test01.py e test11.py gi� scritti all'interno del modulo e funzionanti. \\

\noindent Nel test01 vengono testate le funzioni \texttt{csrCreate} e figlie, \texttt{csr2DenseMatrix} e figlie, \texttt{matrixProduct}, 
\texttt{csrTranspose}; nel test11 si testano invece \texttt{larFacets} e figlie, \texttt{mkSignedEdges} e \texttt{larModelNumbering}.\\

\noindent Gli stessi test possono essere utilizzati per verificare il corretto funzionamento delle versioni parallelizzate delle funzioni, 
semplicemente invocando le funzioni parallelizzate al posto di quelle seriali. 

\subsection{test01}

\noindent \textbf{Python}

\begin{codice}
""" Model generation, skeleton and boundary extraction """
from larlib import *

# input of geometry and topology  
V2 = [[4,10],[8,10],[14,10],[8,7],[14,7],[4,4],[8,4],[14,4]]
EV = [[0,1],[1,2],[3,4],[5,6],[6,7],[0,5],[1,3],[2,4],[3,6],[4,7]]
FV = [[0,1,3,5,6],[1,2,3,4],[3,4,6,7]]

# characteristic matrices
csrFV = csrCreate(FV)
csrEV = csrCreate(EV)
# print "\nFV =\n", csr2DenseMatrix(csrFV)
print("\nFV =\n", csr2DenseMatrix(csrFV))
# print "\nFV =\n", csr2DenseMatrix(csrFV)
print("\nEV =\n", csr2DenseMatrix(csrEV))

# product
csrEF = matrixProduct(csrEV, csrTranspose(csrFV))
# print "\nEF =\n", csr2DenseMatrix(csrEF)
print("\nEF =\n", csr2DenseMatrix(csrEF))

# boundary and coboundary operators
facetLengths = [csrCell.getnnz() for csrCell in csrEV]
boundary = csrBoundaryFilter(csrEF,facetLengths)
coboundary_1 = csrTranspose(boundary)
# print "\ncoboundary_1 =\n", csr2DenseMatrix(coboundary_1)
print("\ncoboundary_1 =\n", csr2DenseMatrix(coboundary_1))

# product operator
mod_2D = (V2,FV)
V1,topol_0 = [[0.],[1.],[2.]], [[0],[1],[2]]
topol_1 = [[0,1],[1,2]]
mod_0D = (V1,topol_0)
mod_1D = (V1,topol_1)
V3,CV = larModelProduct([mod_2D,mod_1D])
mod_3D = (V3,CV)
VIEW(EXPLODE(1.2,1.2,1.2)(MKPOLS(mod_3D)))
# print "\nk_3 =", len(CV), "\n"
print("\nk_3 =", len(CV), "\n")

# 2-skeleton of the 3D product complex
mod_2D_1 = (V2,EV)
mod_3D_h2 = larModelProduct([mod_2D,mod_0D])
mod_3D_v2 = larModelProduct([mod_2D_1,mod_1D])
_,FV_h = mod_3D_h2
_,FV_v = mod_3D_v2
FV3 = FV_h + FV_v
SK2 = (V3,FV3)
VIEW(EXPLODE(1.2,1.2,1.2)(MKPOLS(SK2)))
# print "\nk_2 =", len(FV3), "\n"
print("\nk_2 =", len(FV3), "\n")

# 1-skeleton of the 3D product complex 
mod_2D_0 = (V2,AA(LIST)(range(len(V2))))
mod_3D_h1 = larModelProduct([mod_2D_1,mod_0D])
mod_3D_v1 = larModelProduct([mod_2D_0,mod_1D])
_,EV_h = mod_3D_h1
_,EV_v = mod_3D_v1
EV3 = EV_h + EV_v
SK1 = (V3,EV3)
VIEW(EXPLODE(1.2,1.2,1.2)(MKPOLS(SK1)))
# print "\nk_1 =", len(EV3), "\n"
print("\nk_1 =", len(EV3), "\n")

# boundary and coboundary operators
np.set_printoptions(threshold=sys.maxint)
csrFV3 = csrCreate(FV3)
csrEV3 = csrCreate(EV3)
csrVE3 = csrTranspose(csrEV3)
facetLengths = [csrCell.getnnz() for csrCell in csrEV3]
boundary = csrBoundaryFilter(csrVE3,facetLengths)
coboundary_0 = csrTranspose(boundary)
# print "\ncoboundary_0 =\n", csr2DenseMatrix(coboundary_0)
print("\ncoboundary_0 =\n", csr2DenseMatrix(coboundary_0))

csrEF3 = matrixProduct(csrEV3, csrTranspose(csrFV3))
facetLengths = [csrCell.getnnz() for csrCell in csrFV3]
boundary = csrBoundaryFilter(csrEF3,facetLengths)
coboundary_1 = csrTranspose(boundary)
# print "\ncoboundary_1.T =\n", csr2DenseMatrix(coboundary_1.T)
print("\ncoboundary_1.T =\n", csr2DenseMatrix(coboundary_1.T))

csrCV = csrCreate(CV)
csrFC3 = matrixProduct(csrFV3, csrTranspose(csrCV))
facetLengths = [csrCell.getnnz() for csrCell in csrCV]
boundary = csrBoundaryFilter(csrFC3,facetLengths)
coboundary_2 = csrTranspose(boundary)
# print "\ncoboundary_2 =\n", csr2DenseMatrix(coboundary_2)
print("\ncoboundary_2 =\n", csr2DenseMatrix(coboundary_2))

# boundary chain visualisation
boundaryCells_2 = boundaryCells(CV,FV3)
boundary = (V3,[FV3[k] for k in boundaryCells_2])
VIEW(EXPLODE(1.5,1.5,1.5)(MKPOLS(boundary)))
\end{codice}

\vspace{0.5cm}

\noindent \textbf{Julia}

\begin{codice}
using PyCall
using DataStructures

@pyimport larlib as l
@pyimport numpy as np
@pyimport pyplasm as p
@pyimport scipy.sparse as ss
@pyimport sys

include("traduzione.jl")
include("largrid.jl")

# input of geometry and topology  
V2 = [[4,10],[8,10],[14,10],[8,7],[14,7],[4,4],[8,4],[14,4]]
EV = [[0,1],[1,2],[3,4],[5,6],[6,7],[0,5],[1,3],[2,4],[3,6],[4,7]]
FV = [[0,1,3,5,6],[1,2,3,4],[3,4,6,7]]

# characteristic matrices
csrFV = csrCreate(FV)
csrEV = csrCreate(EV)
println("FV = ", csr2DenseMatrix(csrFV), "\n")
println("EV = ", csr2DenseMatrix(csrEV), "\n")

# product
csrEF = matrixProduct(csrEV, csrTranspose(csrFV))
println("EF = ", csr2DenseMatrix(csrEF), "\n")

# boundary and coboundary operators
facetLengths = []
for csrCell in csrEV
    facetLengths = vcat(facetLengths, csrCell[:getnnz]())
end
boundary = l.csrBoundaryFilter(csrEF,facetLengths)
coboundary_1 = csrTranspose(boundary)
println("coboundary_1 = ", csr2DenseMatrix(coboundary_1), "\n")

# product operator
mod_2D = V2,FV
V1 = [[0.],[1.],[2.]]
topol_0 = [[0],[1],[2]]
topol_1 = [[0,1],[1,2]]
mod_0D = V1,topol_0
mod_1D = V1,topol_1
V3,CV = larModelProduct([mod_2D,mod_1D])
larExplodedView(V3,CV)
println("\nk_3 = ", length(CV), "\n")

# 2-skeleton of the 3D product complex
mod_2D_1 = V2,EV
mod_3D_h2 = larModelProduct([mod_2D,mod_0D])
mod_3D_v2 = larModelProduct([mod_2D_1,mod_1D])
_,FV_h = mod_3D_h2
_,FV_v = mod_3D_v2
FV3 = vcat(FV_h, FV_v)
larExplodedView(V3,FV3)
println("\nk_2 = ", length(FV3), "\n")

# 1-skeleton of the 3D product complex 
list = []
for i in collect(0:length(V2)-1)
    list = vcat(list, [[i]])
end 
mod_2D_0 = V2,list
mod_3D_h1 = larModelProduct([mod_2D_1,mod_0D])
mod_3D_v1 = larModelProduct([mod_2D_0,mod_1D])
_,EV_h = mod_3D_h1
_,EV_v = mod_3D_v1
EV3 = vcat(EV_h, EV_v)
larExplodedView(V3,EV3)
println("\nk_1 = ", length(EV3), "\n")

# boundary and coboundary operators
np.set_printoptions(threshold=sys.maxint)
csrFV3 = csrCreate(FV3)
csrEV3 = csrCreate(EV3)
csrVE3 = csrTranspose(csrEV3)
facetLengths = []
for csrCell in csrEV3
    facetLengths = vcat(facetLengths, csrCell[:getnnz]())
end

csrEF3 = matrixProduct(csrEV3, csrTranspose(csrFV3))
facetLengths = []
for csrCell in csrFV3
    facetLengths = vcat(facetLengths, csrCell[:getnnz]())
end
boundary = l.csrBoundaryFilter(csrEF3,facetLengths)
coboundary_1 = csrTranspose(boundary)
println("coboundary_1.T = ", csr2DenseMatrix(coboundary_1[:transpose]()), "\n")

csrCV = csrCreate(CV)
csrFC3 = matrixProduct(csrFV3, csrTranspose(csrCV))
facetLengths = []
for csrCell in csrCV
    facetLengths = vcat(facetLengths, csrCell[:getnnz]())
end
boundary = l.csrBoundaryFilter(csrFC3,facetLengths)
coboundary_2 = csrTranspose(boundary)
println("coboundary_2 = ", csr2DenseMatrix(coboundary_2), "\n")

# boundary chain visualisation
boundaryCells_2 = l.boundaryCells(CV,FV3)
larExplodedView(V3, [FV3[k+1] for k in boundaryCells_2])
\end{codice}

\vspace{0.5 cm}

\begin{figure}[H]
  \centering
  %\subfloat[][PF1]
  {\includegraphics[width=.48\columnwidth]{immagini/test01_1}} \quad 
  %\subfloat[][PF2]
  {\includegraphics[width=.48\columnwidth]{immagini/test01_2}} \\
  \vspace{0.5 cm}
  %\subfloat[][PF3]
  {\includegraphics[width=.48\columnwidth]{immagini/test01_3}} \quad
  %\subfloat[][PF4]
  {\includegraphics[width=.48\columnwidth]{immagini/test01_4}} \\
  \caption{Risultati grafici durante l'esecuzione di \texttt{test01.jl}}
\label{fig:subfig}
\end{figure}


\subsection{test11}

\noindent \textbf{Python}

\begin{codice}
""" Example of oriented edge drawing """
from larlib import *

V = [[9,0],[13,2],[15,4],[17,8],[14,9],[13,10],[11,11],[9,10],[7,9],[5,9],[3,
8],[0,6],[2,3],[2,1],[5,0],[7,1],[4,2],[12,10],[6,3],[8,3],[3,5],[5,5],[7,6],
[8,5],[10,5],[11,4],[10,2],[13,4],[14,6],[13,7],[11,9],[9,7],[7,7],[4,7],[2,
6],[12,7],[12,5]]

FV = [[0,1,26],[5,6,17],[6,7,17,30],[7,30,31],[7,8,31,32],[24,30,31,35],[3,4,
28],[4,5,17,29,30,35],[4,28,29],[28,29,35,36],[8,9,32,33],[9,10,33],[11,10,
33,34],[11,20,34],[20,33,34],[20,21,32,33],[18,21,22],[21,22,32],[22,23,31,
32],[23,24,31],[11,12,20],[12,16,18,20,21],[18,22,23],[18,19,23],[19,23,24],
[15,19,24,26],[0,15,26],[24,25,26],[24,25,35,36],[2,3,28],[1,2,27,28],[12,13,
16],[13,14,16],[14,15,16,18,19],[1,25,26,27],[25,27,36],[36,27,28]]

VIEW(EXPLODE(1.2,1.2,1)(MKPOLS((V,FV))))
VV = AA(LIST)(range(len(V)))
_,EV = larFacets((V,FV+[range(16)]),dim=2,emptyCellNumber=1)

submodel = mkSignedEdges((V,EV))
VIEW(submodel)
VIEW(larModelNumbering(scalx=1,scaly=1,scalz=1)(V,[VV,EV,FV],submodel,2))
\end{codice}

\vspace{0.5cm}

\noindent \textbf{Julia}

\begin{codice}
using PyCall
using DataStructures

@pyimport larlib as l
@pyimport numpy as np
@pyimport pyplasm as p
@pyimport scipy.sparse as ss
@pyimport sys

include("traduzione.jl")
include("largrid.jl")

V = [[9,0],[13,2],[15,4],[17,8],[14,9],[13,10],[11,11],[9,10],[7,9],[5,9],[3,
8],[0,6],[2,3],[2,1],[5,0],[7,1],[4,2],[12,10],[6,3],[8,3],[3,5],[5,5],[7,6],
[8,5],[10,5],[11,4],[10,2],[13,4],[14,6],[13,7],[11,9],[9,7],[7,7],[4,7],[2,
6],[12,7],[12,5]]

FV = [[0,1,26],[5,6,17],[6,7,17,30],[7,30,31],[7,8,31,32],[24,30,31,35],[3,4,
28],[4,5,17,29,30,35],[4,28,29],[28,29,35,36],[8,9,32,33],[9,10,33],[11,10,
33,34],[11,20,34],[20,33,34],[20,21,32,33],[18,21,22],[21,22,32],[22,23,31,
32],[23,24,31],[11,12,20],[12,16,18,20,21],[18,22,23],[18,19,23],[19,23,24],
[15,19,24,26],[0,15,26],[24,25,26],[24,25,35,36],[2,3,28],[1,2,27,28],[12,13,
16],[13,14,16],[14,15,16,18,19],[1,25,26,27],[25,27,36],[36,27,28]]

larExplodedView(V,FV)
VV = [] 
for i in collect(0:length(V)-1)
    VV = vcat(VV, [[i]])
end 
model = (V, vcat(FV,[collect(0:15)]))
_,EV = plarFacets(model,2,1)

submodel = pmkSignedEdges((V,EV))
p.VIEW(submodel)
\end{codice}

\vspace{0.5 cm}

\begin{figure}[H]
  \centering
  %\subfloat[][PF1]
  {\includegraphics[width=.48\columnwidth]{immagini/test11_1}} \quad 
  %\subfloat[][PF2]
  {\includegraphics[width=.48\columnwidth]{immagini/test11_2}} \\
  \vspace{0.5 cm}
  %\subfloat[][PF3]
  {\includegraphics[width=1\columnwidth]{immagini/test11_3b}} \\
  \caption{Risultati grafici durante l'esecuzione di \texttt{test11.jl}}
\label{fig:subfig}
\end{figure}