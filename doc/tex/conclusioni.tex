\singlespacing

\chapter{Conclusioni}

Dallo svolgimento di diversi test sul numero di processori, si nota che la parallelizzazione di alcune delle funzioni rallenta l'esecuzione delle 
stesse. Un altro tentativo � stato effettuato andando ad agire su diversi input di grandi dimensioni. Sono state pertanto confrontate le performance 
delle funzioni parallelizzate con quelle seriali, utilizzando 15 processori, su diversi input di grandi dimensioni. \\

\noindent Segue lo studio sulla funzione \texttt{csrCreate} con diversi input con n = 15. \\

\begin{codice}
input1 = vcat([EV*j for j=1:100]...) 
input2 = vcat([EV*j for j=1:1000]...)
input3 = vcat([EV*j for j=1:5000]...)
\end{codice}

\noindent dove 

\begin{codice}
EV = [[0,1],[0,3],[1,2],[1,3],[1,4],[2,4],[2,5],[3,4],[4,5]] 
\end{codice}

\vspace{0.5 cm}

\noindent \textbf{Tempi della versione seriale}
\begin{codice}
timeinput1 = 0.002113414
timeinput2 = 0.008444055
timeinput3 = 0.050567467
\end{codice}

\vspace{0.5 cm}

\noindent \textbf{Tempi della versione parallela}
\begin{codice}
timepinput1 = 0.002388622
timepinput2 = 0.00893245
timepinput3 = 0.060305142
\end{codice}

\vspace{0.5 cm}

\noindent \textbf{Confronti tra la versione seriale e la parallela tramite il rapporto tra i loro tempi}
\begin{codice}
timeinput1/timepinput1 = 0.8847837790994139
timeinput2/timepinput2 = 0.945323511466619
timeinput3/timepinput3 = 0.8385266218260459
\end{codice}

\vspace{0.5 cm}

\noindent Come si evince dall'esempio di cui sopra, la versione parallela non � efficiente. \\\\

\noindent Esempio sicuramente pi� significativo � dato dallo studio sulla funzione \texttt{csr2DenseMatrix}. \\

\begin{codice}
input1 = vcat([EV*j for j=1:100]...)
input2 = vcat([EV*j for j=1:1000]...)
input3 = vcat([EV*j for j=1:5000]...)
\end{codice}

\vspace{0.5 cm}

\noindent \textbf{Tempi della versione seriale}
\begin{codice}
timeinput1 = 0.009490412
timeinput2 = 0.466812199
timeinput3 = 12.684459215
\end{codice}

\vspace{0.5 cm}

\noindent \textbf{Tempi della versione parallela}
\begin{codice}
timepinput1 = 0.010543615
timepinput2 = 0.669056614
timepinput3 = 14.3567615
\end{codice}

\vspace{0.5 cm}

\noindent \textbf{Confronti tra la versione seriale e la parallela tramite il rapporto tra i loro tempi}
\begin{codice}
timeinput1/timepinput1 = 0.9001098769255138
timeinput2/timepinput2 = 0.6977170380382787
timeinput3/timepinput3 = 0.8835181398674068	
\end{codice}

\vspace{0.5 cm}

\noindent Si pu� dunque concludere che la versione seriale, in caso di grandi input, aiuta a velocizzare il processo.